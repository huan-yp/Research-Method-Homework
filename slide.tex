\documentclass{beamer}
\usepackage{ctex, hyperref}
\usepackage[T1]{fontenc}

% other packages
\usepackage{latexsym,xcolor,multicol,booktabs,calligra}
\usepackage{amsmath,amssymb,amsfonts}
\usepackage{graphicx,pstricks,listings,stackengine}
\usepackage{float}
\usepackage{subfig}
\usepackage{algorithmic}
\usepackage{algorithm}
\usepackage{multirow}

\author{学生:毛周祥~~~指导老师:单冯}
\title{无标注预训练方法在视觉语言模型的应用}
\subtitle{科学研究方法--课程作业}
\institute{东南大学计算机拔尖培养基地}
\date{2025 年 9 月}
\usepackage{SEU}

% defs
\def\cmd#1{\texttt{\color{red}\footnotesize $\backslash$#1}}
\def\env#1{\texttt{\color{blue}\footnotesize #1}}
\definecolor{seu-green}{rgb}{0.345,0.459,0.345}      % #587558
\definecolor{seu-yellow}{rgb}{0.992,0.816,0}         % #fdd000
\definecolor{seu-black}{rgb}{0.137,0.094,0.082}      % #231815
\definecolor{deepblue}{rgb}{0,0,0.5}
\definecolor{deepred}{rgb}{0.6,0,0}
\setbeamerfont{normal text}{size=\Large}
\setbeamerfont{sub normal text}{size=\large}

% 导言区
\makeatletter
\newenvironment{Litemize}{%
  \par\begingroup
  \def\beamer@setuplines{\baselineskip=36pt\relax}% 行距
  \usebeamerfont{itemize item}% 让 beamer 保留标签设置
  \Large\linespread{1}\selectfont      % 字号 + 行距
  \begin{itemize}[<+->]
}{%
  \end{itemize}%
  \endgroup
}

\newcommand{\subtit}[1]{\\[2pt]  % 换行+微小垂直间隙
   {\normalsize\color{gray!70} #1 \par}}   % 灰色、小一号


\newenvironment{litemize}{%
  \par\begingroup
  \def\beamer@setuplines{\baselineskip=12pt\relax}%
  \usebeamerfont{itemize item}%
  \large\linespread{1.25}\selectfont
  \begin{itemize}[<+->]
}{%
  \end{itemize}%
  \endgroup
}

\newenvironment{litemize-p}{%
  \par\begingroup
  \def\beamer@setuplines{\baselineskip=12pt\relax}%
  \usebeamerfont{itemize item}%
  \large\linespread{1.25}\selectfont
  \begin{itemize}[]
}{%
  \end{itemize}%
  \endgroup
}
\makeatother

\lstset{
    basicstyle=\ttfamily\small,
    keywordstyle=\bfseries\color{deepblue},
    emphstyle=\ttfamily\color{deepred},    % Custom highlighting style
    stringstyle=\color{seu-green},
    numbers=left,
    numberstyle=\small\color{seu-black},
    rulesepcolor=\color{red!20!green!20!blue!20},
    frame=shadowbox,
}


\begin{document}


\kaishu
\begin{frame}
    \titlepage
    \begin{figure}[htpb]
        \begin{center}
            \includegraphics[width=0.33\linewidth]{pic/SEU-09J.png}
        \end{center}
    \end{figure}
\end{frame}

\begin{frame}
    \tableofcontents[sectionstyle=show,subsectionstyle=show/shaded/hide,subsubsectionstyle=show/shaded/hide]
\end{frame}

\section{研究背景}

\begin{frame}


    \frametitle{为什么需要视觉语言模型 (VLM)?}
    \begin{columns}
        \column{0.5\textwidth}
        \begin{Litemize}
            \item 一句话定义 VLM
            \item 它能做什么? 
                \begin{litemize-p}
                    \item 图文配对理解
                    \item 视觉问答(VQA)
                    \item 图像字幕生成
                \end{litemize-p}
        \end{Litemize}
        \column{0.5\textwidth}
        \begin{figure}[htpb]
            \begin{center}
                \includegraphics<3->[width=1\linewidth]{pic/VLM-usage.png}
            \end{center}
        \end{figure}
    \end{columns}
    
\end{frame}

\begin{frame}
    \frametitle{无标注预训练方法的 Motivation}
    \begin{columns}
        \column{0.5\textwidth}
        \begin{Litemize}
            \item 现有方法的局限
            \begin{litemize-p}
                \item 依赖昂贵的有标注数据
                \item 难以扩展到新任务
                \item 图像区域与文本词汇难以精细对齐
            \end{litemize-p}
            \item NLP 领域的成功经验
            \begin{litemize-p}
                \item 从 Word2Vec 到 GPT
                \item 自监督预训练
            \end{litemize-p}      
        \end{Litemize}
        \column{0.5\textwidth}
        \begin{figure}[htpb]
            \begin{center}
                \includegraphics<1->[width=1\linewidth]{pic/why-pretraining-1.png}
            \end{center}
        \end{figure}
        \begin{figure}[htpb]
            \begin{center}
                \includegraphics<5->[width=1\linewidth]{pic/why-pretraining-2.png}
            \end{center}
        \end{figure}
    \end{columns}
\end{frame}

\begin{frame}
    \frametitle{大规模无标注预训练 \\
    ~~~~~~----- 观千剑而后识器,操千曲而后晓声
    }
    \begin{columns}
        \column{0.5\textwidth}
        \begin{Litemize}
            \item 突破标注瓶颈
            \item \color{seu-yellow}零样本泛化\color{black}涌现
            \item 统一表示空间
            \item 通用下游任务范式
        \end{Litemize}
        \column{0.5\textwidth}
        \begin{figure}[htpb]
            \begin{center}
                \includegraphics<4->[width=1\linewidth]{pic/VLM-pretraining-workflow.png}
            \end{center}
        \end{figure}
    \end{columns}
\end{frame}

\section{CLIP}
\begin{frame}
    \frametitle{CLIP (2021)}
    \begin{columns}
        \column{0.6\textwidth}
        \begin{litemize}
            \item 目标:学习图文的联合表示
            \item 方法:对比学习
            \item 数据集:4 亿对图文数据
            \item 结果:多种下游任务表现优异
        \end{litemize}
        \column{0.5\textwidth}
        \begin{figure}[htpb]
            \begin{center}
                \includegraphics<1->[width=1\linewidth]{pic/CLIP-demo.png}
            \end{center}
        \end{figure}
    \end{columns}
\end{frame}

\begin{frame}
    \frametitle{CLIP 的架构}
    \begin{columns}
        \column{0.5\textwidth}
        \begin{litemize}
            \item 图像编码器:ResNet 或 ViT
            \item 文本编码器:Text Transformer
            \item 总结构:双塔架构
        \end{litemize}
        \column{0.5\textwidth}
        \begin{figure}[htpb]
            \begin{center}
                \includegraphics<1>[width=1.2\linewidth]{pic/CLIP-architecture-3.png}
            \end{center}
        \end{figure}
        \begin{figure}[htpb]
            \begin{center}
                \includegraphics<2>[width=1\linewidth]{pic/CLIP-architecture-2.png}
            \end{center}
        \end{figure}
        \begin{figure}[htpb]
            \begin{center}
                \includegraphics<3>[width=1\linewidth]{pic/CLIP-architecture-1.png}
            \end{center}
        \end{figure}
    \end{columns}
\end{frame}
\begin{frame}
    \frametitle{CLIP 的训练——对比学习}
    \begin{columns}
        \column{0.5\textwidth}
        \begin{litemize}
            \item 训练方法:图文对比 (ITC)
            \item 优化函数:对比损失
            \item 为什么有效?
            \begin{litemize-p}
                \item 损失函数就是考试成绩
                \item 倒逼模型学习一个合理的图文表示
                \item 不相关的推远,相关的拉近
                \item 大规模数据+大模型
            \end{litemize-p}
        \end{litemize}
        \column{0.5\textwidth}
        \begin{figure}[htpb]
            \begin{center}
                \includegraphics<1->[width=1\linewidth]{pic/CLIP-ITC.png}
            \end{center}
        \end{figure}
    \end{columns}
\end{frame}
\begin{frame}
    \frametitle{CLIP 的训练——队列}
    \begin{columns}
        \column{0.5\textwidth}
        \begin{litemize}
            \item 四亿张图需要两两对比?
            \item 批次大小受限,负例不足。
            \item 解决方案:队列 (Queue)
        \end{litemize}
        \begin{figure}
            \begin{center}
                \includegraphics<4->[width=0.7\linewidth]{pic/GM-contrast.png}
            \end{center}
        \end{figure}
        \column{0.5\textwidth}
        \begin{figure}[htpb]
            \begin{center}
                \includegraphics<1->[width=1\linewidth]{pic/CLIP-ITC.png}
            \end{center}
        \end{figure}
    \end{columns}
\end{frame}
\begin{frame}
    \frametitle{CLIP 的下游使用}
    \begin{columns}
        \column{0.5\textwidth}
        \begin{litemize}
            \item ZeroShot 图像分类。
            \item 作为其他任务的编码器。\\
            eg. Stable Diffusion
            \item 为什么能迁移?
            \begin{litemize-p}
                \item 语言本身就是监督信号
                \item 只有学习到如何表示图片特征,才能完成预训练任务 \\
                只有学会了课上讲的内容,才能独立完成实验作业
            \end{litemize-p}
        \end{litemize}

        \column{0.5\textwidth}
        \begin{figure}[htpb]
            \begin{center}
                \includegraphics<1->[width=1\linewidth]{pic/CLIP-zeroshot.png}
            \end{center}
        \end{figure}
    \end{columns}
\end{frame}

\section{ALBEF}

\begin{frame}
    \frametitle{CLIP 的缺陷}
    \begin{columns}
        \column{0.5\textwidth}
        \begin{litemize}
            \item 细粒度对齐能力不足
            \begin{litemize-p}
                \item 训练时只针对图文整体
                \item 无法精确匹配图像区域与文本词汇
                \item 难以处理复杂场景
            \end{litemize-p}
            \item 数据集质量参差不齐
            \begin{litemize-p}
                \item 网络爬取,噪声多
                \item 不准确标签极大影响模型学习
            \end{litemize-p}
        \end{litemize}
        \column{0.5\textwidth}
        \begin{figure}[htpb]
            \begin{center}
                % \includegraphics<1->[width=1\linewidth]{pic/ALBEF-model.png}
            \end{center}
        \end{figure}
    \end{columns}
\end{frame}
\begin{frame}
    \frametitle{ALBEF (2021)}
\end{frame}
\begin{frame}
    \frametitle{ITM + MLM}
\end{frame}
\begin{frame}
    \frametitle{动量蒸馏 (MoD)}
\end{frame}
\section{BLIP}
\begin{frame}
    \frametitle{BLIP (2022)}
\end{frame}
\begin{frame}
    \frametitle{LM 建模}
\end{frame}
\begin{frame}
    \frametitle{数据自举 (Boostraping)}
\end{frame}
\begin{frame}
    \frametitle{性能对比}
\end{frame}
\section{拓展介绍}
\begin{frame}
    \frametitle{下游泛化——prompt tuning (2023)}
\end{frame}
\begin{frame}
    \frametitle{下游泛化——Text as Image (2023)}
\end{frame}
\begin{frame}
    \frametitle{应用生态}
\end{frame}

\section{最后}
\begin{frame}
    \frametitle{总结}
    \begin{itemize}
        \item VLM 及其预训练方法的重要性
        \item 为什么预训练任务有效
        \begin{litemize-p}
            \item 预训练任务是数学考试
            \item 训练目标是考高分、倒逼模型学好特征
            \item 学好特征,更好的迁移到下游任务
        \end{litemize-p}
        \item VLM 的演进
        \begin{litemize-p}
            \item CLIP:对比学习,双塔架构
            \item ALBEF:细粒度对齐尝试,动量蒸馏
            \item BLIP:数据自举
        \end{litemize-p}
    \end{itemize}
\end{frame}

\begin{frame}
    \frametitle{Q \& A}
\end{frame}

\begin{frame}
    \begin{center}
        {\Huge\calligra Thanks!}
    \end{center}
\end{frame}

\end{document}